\chapter{Zusammenfassung}
Die Ziele im Rahmen des Projektes wurden erreicht. Das heißt, es wurde erfolgreich eine Simulationseinheit umgesetzt, die sowohl software- als auch hardwareseitig den Anforderungen entspricht. Sie kann auf einem Raspberry PI installiert in eine Kühlanlage integriert werden und somit Daten für den Stromverbrauch eines Eiswasserspeichers sowie dessen Einfluss auf ein solches System erstellen.
Mithilfe des Steuer-Servers können Kältemaschine für das Laden und Kreiselpumpe für das Kühlen unabhängig voneinander über das Internet bedient, d.h. an- und ausgeschaltet werden. Der Koordinator sorgt für den zeitdiskreten Ablauf und der Speicher führt das zugrunde liegende physikalische Modell aus.
Es ist im Moment nicht möglich zu beurteilen, wie realistisch die Daten sind, die der Simulator generiert, da der direkte Vergleich mit einem realen System im Rahmen des Projektes nicht möglich war.

\section{Ausblick}
Es gibt noch verschiedene Möglichkeiten, weitere Erweiterungen in den Simulator einzubauen. In der aktuellen Verfassung kann beispielsweise nicht entschieden werden, ob die Vakuumpumpen an- oder ausgeschaltet sind. Es wird somit davon ausgegangen, dass diese dauerhaft an sind, was das Ergebnis leicht verfälschen könnte. Weiterhin ist das Modell im Moment noch sehr ungenau, da viele Parameter und Variablen als konstant angesehen werden. So könnte man unter anderem den Stromverbrauch des Eiswasserspeichers mit einem realistischeren Modell berechnen oder die Außentemperatur mit in die Berechnungen einbeziehen. % Keine Ahnung ^^